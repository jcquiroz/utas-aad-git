%!TEX root = ../jcq-text-v1.tex
\section*{\scshape Chapter 4}
{\Large \scshape Bio-economic management strategy evaluation of \\
 \acl{top} (\emph{Dissostichus eleginoides}) in  southern \\ and antarctic oceans} % \\[0.1\baselineskip] 
\label{sec:chapter4}

\subsection*{Rationale}

Previous studies highlight that \acl{top} is vulnerable to exploitation and make up one of the most major and valuable fishery in southern and antarctic regions \citep{Constable2000778,Norse2012}. Therefore, fisheries management based on robust scientific knowledge should be highly desirable in southern and antarctic exploitation area, even more when modelling and management of \acl{top} population has added complexity related with population shared between nations \citep{Candy2011b}. 

Since the \acl{top} is managed by harvest strategy and control rules in CCAMLR's areas, and also explicit management objectives and reference points (e.i. bound to define harvest rules) have been recently proposed in South-America, the aim in this chapter is evaluate the robustness of these management procedures to a wide range of uncertainties, particularly the model error related to uncertainty about the spatial structure of \acl{top} population. 

A second objective should be evaluate alternative or candidate management procedures which using optimal pathways try to accomplish the management objectives. Several research have shown that economic dimension is crucial to an adequate management system \citep{Dichmont2010,calebthesis,Hilborn2012,eltit,Hoshino2014,Punt2013116,Emery2014}. Therefore, the management optimal pathways proposed in this chapter should include economic aspects together with the current biological and ecological dimension.


\subsection*{Methods}

In this chapter, a Management Strategy Evaluation (MSE) is proposed  to evaluate the consequences of alternative hypotheses in terms of achieving management objectives. The MSE is a simulation-base tool and has been applied in several fisheries, providing an appropriate simulation framework for evaluate management strategies and also compare alternative management options \citep{Aranda,Milner-Gulland2011,Hilborn2012}. 

Since the MSE approach is wider in scope, for example, multiple evaluation criteria could be applied to process error, observation error, estimation error and implementation of harvest strategies error, the mathematical and statistical methods developed in this chapter should be bounded to outcomes of chapters 2 \& 3 (see pag.~\pageref{sec:chapter2} and \pageref{sec:chapter3}) and the findings from embedded project indicated in preceding chapters (see pag.~\pageref{subsec:extsupp}).


\subsection*{Outcomes intended}

Publishing type: \textit{Peer-reviewed article} \\
Target journal: \textit{ICES Journal of Marine Science (Oxford University Press, UK)} \\
Objectives addressed: \textit{Objectives 3 \& 4, see page~\pageref{subsec:objetives} }\\
Main milestone: \textit{Manuscript submission by January,  2017}


\subsection*{Threat \& Contingency}

Because the MSE application depend of outcomes from chapters 2 \& 3 as well as the embedded project, is impossible give major details about the candidates strategies or comparison methods through exploitation areas. Therefore, the MSE scope under this chapter could vary substantially.   


\vspace{1.8cm}
