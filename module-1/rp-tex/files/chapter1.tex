%!TEX root = ../jcq-text-v1.tex
\section*{\scshape Chapter 1}
{\Large \scshape \acl{top} fishery in South-America: \\ Drawing lesson from other Toothfish fisheries.} % \\[0.1\baselineskip] 

\subsection*{Rationale}

The \acl{top} fishery on South-America began late 1970's, similar to the main exploitation areas of \ac{ccamlr} \citep{FAO2014,ccamlr90}. However, fisheries management elements on South-America such as management objectives, harvest rules, precautionary/ecosystem approach, implementation process, transparency, stakeholders involved and stewardship levels by quota owner are almost unknown to most of scientific community. A comprehensive search on Web of Knowledge$^{\copyright}$ using key topics management/toothfish + Chile/Argentine yields only 3 published sources,  increasing to 15 sources when the search include Latin American repositories.

The lack of a knowledge base for fisheries management of \acl{top} on South-America shows two broad themes. First, any attempts to evaluate alternative management options (like those proposal on Chapter 4, see pag.~\pageref{sec:chapter4}) require identify and quantify the stated management objectives \citep{Aranda,Deroba2008,Milner-Gulland2011}. Second, in the light of large and complex population dynamics such as \acl{top} (\emph{e.g.} HIMI and kerguelen Islands; Southwest Atlantic and Southwest Pacific oceans in South-America), the use the data from different jurisdictions to implement robust population modelling (like those proposal on Chapter 2 \& 3) require identify explicitly the harvest rules \citep{Constable2000778}.

Therefore, a detailed review and comparison between the Kerguelen Plateau and South-America's management approaches should overcome this lack of knowledge base and also give support the following chapters of this PhD research, principally those related to improve the mathematical and statistical methods, as well as the management procedures.


\subsection*{Methods}

From history of fisheries assessment methods, implemented models and management procedures both in South-America and Kerguelen Plateau areas, a time-line of features and actions should be constructed to characterizing and comparing weakest and successful attributes of both areas. Data of management actions and implementation of assessment methods along of fishery history should be obtained from follows sources:

{\small 
\begin{itemize}
	\item Chile:
	\begin{itemize}
		\item Undersecretary of Fisheries
		\item Fisheries Research Institute 
		\item National Fishing Service
		\item Fishing Scientific Committees   
	\end{itemize}
	\item Argentine:
	\begin{itemize}
		\item Undersecretary of Fisheries and Aquaculture
		\item National Institute for Development and Research in Fisheries
	\end{itemize}
		\item Kerguelen Plateau:
	\begin{itemize}
		\item Report and conservations measures from \ac{ccamlr}
		\item Scientific bibliography
	\end{itemize}
\end{itemize}
}

In the case of South-America, most of the data come from different sources and encompass unpublished technical reports, meeting minutes, and digital database. The Chilean management has been framed since 1990 by ``The General Act of Fishing and Aquaculture'', and between 2001-2012 a system for catch allocation was based on the rationale of individual transferable quota system (ITQ). However, this system lacked of explicit management objectives and a specific procedure to setting total allowable catches (TAC) using RPs. During 2012 several amendments to the General Fishing Act were introduced, the most important are: i) close the access to those fisheries subjected to TAC by allocating the ITQs to small groups of industrial fishermen over at least 40 years and ii) define the maximum sustainable yield (MSY) as the main management objective to quota-based fisheries. Similar scenarios have occurred in Argentine where the cessation of trawl fishery carried out important modifications in longline fleet management. Most of this management aspects of \acl{top} should be discussed and expounded, and also compared with the management process in others \ac{ccamlr} areas.

\subsection*{Outcomes intended}

Publishing type: \textit{Peer-reviewed article} \\
Target journal: \textit{Marine Policy (Elsevier B.V.)} or \textit{Marine Resource Economics (MRE Foundation, Inc.)} \\
Objectives addressed: \textit{Objective 1, see page~\pageref{subsec:objetives} }\\
main milestone: \textit{Manuscript submission by February,  2015}


\subsection*{Threat \& Contingency}
\label{subsec:tconti}

The PhD student already has access to comprehensive unpublished documentation from Chile and also to most of published reports stored on the database of \ac{ccamlr}. However, potential delays or difficulties in obtaining data from Argentina are dependent of an unsigned agreement between Chile and Argentine. This agreement has as objective to develop and implement new methodological approaches of \acl{top} and is scheduled  to be signed no later than December, 2014. 

\vspace{2.5cm}
