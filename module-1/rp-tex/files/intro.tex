%!TEX root = ../jcq-text-v1.tex
\subsection*{Background}

The last decade have been characterized by an increasing debate about the appropriate balance between fishing exploitation and ecological conservation measures required to achieve the sustainability of oceans \citep{Hilborn2013,Worm2009,Worm2013a,Pauly1998}. Conservation biologists and fisheries scientists have engaged in such debate where not always a common ground is achieved. In an recently analysis about global status of fisheries, \citet{Watson2013a} have emphasized that global fishing effort has increased geometrically since 1950s,  supporting the conservationist  arguments that global catch is continuously increasing although with comparatively low rates \citep{Pauly2005,Juan-Jorda2011}, exploited populations worldwide are declining \citep{Myers2003} and  fishery management has been proposed as the main cause of the persistent overexploited status \citep{Worm2006,WormBranch2012}. However, fishery management have been efficient in many harvested populations, especially when economic and social aspects are considered \citep{Hilborn2014}, the mean trophic level trend had increased in many fished populations  \citep{Branch2010,Sethi2010} and several assessed populations, principally in developed countries, have shown a significant recovery in biomass levels \citep{Costello2012}.
% \tablefootnote{Here I define \emph{assessed populations} how these species-populations where a high-quality integrate quantitative assessment have been implemented.}

The deep-sea regions have not been exempt from this debate. Trend of target catches have moved from shallower to deeper water \citep{Morato2006,Watson2013}, which is considered a higher threat to ecological sustainability of extensive deep habitats because it affect the biogeochemical balance of low productive deep environments, but also increases overfishing risks of vulnerable species \citep{Norse2012,Mengerink2014}. However, recent investigations from conservationists and fisheries scientists appear to show a degree of  reconciliation in this debate: it is crucial to develop specie-based management approaches \citep{WormBranch2012,VanDover2014}, where the improvement of assessment methods \citep{Hilborn2014,White2014} is therefore of particular importance. Supporting this points, \citet{Zhou2014} discussed a new paradigm on how fishing effort should be allocated between target and harvested species, recommending new management approaches. An similar guidance was suggested by \citet{Clark2012} to deep-sea fisheries, where the inclusion of spatial dimension on both assessment and management process is crucial to dealing with sustainability of fisheries. Apparently the reconciling key points between these two debate-side are focused on specie-oriented management approach where policies harvest and modelling approaches  should be coherent with the  population dynamics and the food webs where the fishery occurs.

Deep-sea fish species making up the \emph{Dissostichus} genus have been exploited in the southern and antarctic oceans during the last three decades \citep{FAO2014}, and regarding the general debate noted above, also arise opposing investigations about populations distribution, abundance levels, the effectiveness of management and ecosystem health \citep{Constable2000778,csiro2001,eltit2007,Hoshino2010265,Norse2012,Candy2011,Abrams2013,Ainley2013a,Ainley2013,Ziegler2014}. In addition to the philosophical background of this debate (\emph{e.g.} food security, humanity heritage), these investigations reveal three broad issues. Firstly, species belonging the \emph{Dissostichus} genus are presumably the most vulnerable notothenioid species on deep-sea environments, with a high risk for overexploitation due to their demographic traits, namely, large sizes, late maturity and low fertility success \citep[see][]{Collins2010230}. Secondly, despite the management of \emph{Dissostichus} species have been based on ecosystem approaches, at least in the \ac{ccamlr} areas, current harvest rules appears to be simplistic to overcome the adverse ecosystem impact from fishing activities. Finally, the investigations made available a substantial amount of fishery-independent data on \acl{top} (\emph{Dissostichus eleginoides}), which enables to propose modelling research on this specie.

Perhaps the most difficult questions to address the issues indicated above are, where allocate the scientific efforts, into the modelling framework or on harvest policy implementation?, what kind of methods are suitable for \acl{top}?, how the uncertainty transferred to decision-makers could be reduced?, and a further question could be, what impact the improving modelling or management have on conservation of \acl{top}?. The developing of simulation-based Management Strategy Evaluation (MSE) to deal with different uncertainty sources is a promising method to answer such questions \citep{Aranda,Milner-Gulland2011}, but modelling approaches have also been important allowing improve the theoretical dynamic populations behind the quantitative method. For example, \citet{Candy2011b} noted that comprehensive fishery models able to capture the spatial-temporal of \acl{top} require a proper, detailed and deeper data assimilation process to improve the management harvest strategies implemented by Australia and France on Kerguelen Plateau. Likewise, \citet{Ziegler2013} and \citet{wyz2013} pointed out that lack of high quality tagging data, particularly the overlap between range size of tagged fish and those reported on the catches (tag size-overlap), is crucial to estimate robust population size. Despite the need to improve the predictive capacity of \acl{top} fisheries models and provide high-quality science information to decision-makers, theory and implementation underpinning of predictive power on population models is underdeveloped. The lack of effective approaches to accurately characterize process such as population growth derived from length-at-age data, migration patterns based on tagging data, reliable abundance index resulting from standardization of catch per unit effort (CPUE) data and suitable implementation of harvests strategies have preclude the development of comprehensive population models on \acl{top}. 

This PhD research relates to the preceding questions in the context of sustainability process of \acl{top} populations. The aim of this thesis  proposal is to improve both the population modelling and management approaches of \acl{top}, drawing upon the lessons that have been learned in other similar fisheries to develop a MSE tool for understanding the impact of enable and empower predictive future dynamics of \acl{top}. This PhD research will deals with issues related to population inhabiting the following regions: \emph{\textbf{i}}) Kerguelen Plateau, encompassing the areas Heard and McDonald Islands (Australian exclusive economic zone, EEZ) and Kerguelen Islands (French EEZ), \emph{\textbf{ii}}) the continental platform of Argentine EEZ on Southwest Atlantic ocean, and \emph{\textbf{iii}}) over the narrow and deep continental slopes of Chilean EEZ on Southwest Pacific ocean. These regions stand for the fisheries landings of \acl{top} over the \ac{fao} Regions 58, 41 and 87 respectively, and represent yearly around 88\% of official worldwide landings at least during the last 10 years \citep{FAO2014}.

\subsection*{Objetives}
\label{subsec:objetives}

\acl{top} research is a priority area for the \ac{ccamlr} where Australia, France, Argentine and Chile are members. Consistent with the main CCAMLR’ goal of conserving Antarctic marine life, particularly when higher complexities arising related to harvests shared between nations, this PhD research will designed to address the following objectives:

\begin{enumerate}
\item to review, discuss and expound the different \acl{top} management process implemented worldwide 
\item to test the suitability of improve the actual modelling framework of \acl{top} utilized in Kerguelen Plateau and South-America
\item to develop a robust MSE approach of \acl{top} consistent with the modelling improvements identified in Kerguelen Plateau and South-America
\item to examine how the actual harvest policies implemented in Kerguelen Plateau and South-America influences the effectiveness of fishery management on \acl{top}
\end{enumerate}

\subsection*{Thesis outline}
\label{subsec:layout}

The thesis should be structured on a chapters-base as follows:

\textbf{Chapter 1} reviews weakest and successful attributes of management approaches on \acl{top} that have been implemented on southern and antarctic oceans, emphasizing a comparison between Kerguelen Plateau and South-America. 

\textbf{Chapter 2} provides complementary modelling approaches to improve the theoretical population dynamic of \acl{top} on Kerguelen Plateau, particularly related to demographic traits under a spatially-structured base. Fisheries models should play a important role in understanding the fisheries dynamic on Kerguelen Plateau, but there have been few attempts to improve the outcomes from this models by inclusion of demographic traits like migration patterns or spatial structure of population. This Chapter should be a contribution to  recent attempts (see Extensions \& Support section, pag.~\pageref{subsec:extsupp}) to improve the \acl{top} fisheries assessment methods on Kerguelen Plateau.

\textbf{Chapter 3} implements a set of spatially-structured scenarios to explore the implications of omitting spatial demographic structure of \acl{top} on South-America. Recent studies suggest an important connection between  populations that inhabits continental platform of Argentine and slopes of Chile, however a lack on high quality data prevents the implementation of a  spatially-structured empirical model. Using the findings from preceding chapters, the expected results to this chapter should show how significant biases arise due to misspecification model, and also should provide suggestions for further data collection, modelling framework and management strategies.

\textbf{Chapter 4} uses a MSE approach to examine levels of understanding of how changes on implementation of harvest rules influence on success of management actions. The harvest rules defined by \ac{ccamlr} have been designed to avoid significant population reduction and ensure useful removal population consistent with an adequate ecosystem performance. Under an analogous system, but with a lower emphasis to practicing an ecosystem-based management approach, EEZ jurisdictions in South-America have defined different management Reference Points (RPs) to ensure enough escapes and avoid reducing of \acl{top} population. Both in Kerguelen Plateau as South-America, the harvest rules could be implemented using different optimal pathways to accomplish the performance of the management objectives. This Chapter addresses this topic under a spatially simulated MSE approach taken into account the outcomes from Chapter 2 \& 3.

\textbf{Chapter 5} presents a discussion of the thesis’s key findings and conclusions, and suggests functional paths along which future research on \acl{top} might progress.


\subsection*{Extensions \& Support}
\label{subsec:extsupp}
    
This PhD research is embedded within Australian Antarctic Science Strategic Plan implemented by the Australian Antarctic Division (AAD) and should be develop into the framework of project entitled ``Development of Robust Assessment methods and harvest strategies for spatially complex, multi-jurisdictional toothfish fisheries in the Southern Ocean''. The aforementioned project (hereinafter embedded project) is carried out jointly by AAD and the Institute of Marine and Antarctic Science (IMAS) from University of Tasmania. The principal investigator Dr. Dirk Welsford, and Co-Investigators Drs. Klaas Hartmann, Caleb Gardner, Philippe Ziegler, Paul Burch are the central core of this project and also make up  the supervisor team  of this PhD research. The project will be providing a support through the whole PhD research, giving important opportunities to achieve the objectives and guarantees that these are not overdimensioned.

\vspace{5cm}
