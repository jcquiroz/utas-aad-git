%!TEX root = ../jcq-text-v1.tex
\section*{\scshape Chapter 3}
{\Large \scshape Impact of misspecification model under a spatially-structured \\ population, the \acl{top} (\emph{Dissostichus eleginoides}) \\ in South-America} % \\[0.1\baselineskip] 
\label{sec:chapter3}


\subsection*{Rationale}

The \acl{top} tagging programs implemented recently by Chile and Argentine have shown extensive movements of tagged fish released in the continental shelf of Argentine and later recaptured in the Chilean EEZ, as well as from Chile to Argentine \citep{paty2012,trolos2013}. In addition, a south-to-north drift pattern have been persistent from the two years tagging data collected from Chilean commercial vessels \citep{jcq2013}. Furthermore, only one spawning area has been recognized from biological samples and this is split between Argentine and Chile in the far-southern ocean. 

These movement patterns could setting a spatial dynamics similar to what is happening in Kerguelen Plateau. In fact, the principal motivation to extend the current modelling framework in Kerguelen Plateau have been the questions that arising from the tagging data programs regarding the spatial distribution of \acl{top} \citep{Candy2011b}. However, the quality and quantity data obtained from South-America prevents the implementation of a spatially-structured empirical model. Uncertainty about \acl{top} movement also obscure the fleet dynamics, affecting the interpretation of fisheries models. For example, the low percentage of tag size-overlap in Chile EEZ (approx. 46\%) will preclude robust estimates about population size \citep{Candy2011b,wyz2013}. 

Since the similarities between of \acl{top} population dynamic in Kerguelen Plateau and South-America, it seems rational assume that the findings from spatial process on Kerguelen Plateau could be extrapolated to South-America, at least the size-based migration patterns and the quantitative methods to represent this process. The outcomes from chapter 2 (see pag.~\pageref{sec:chapter2}) and the findings from embedded project abovementioned (see pag.~\pageref{subsec:extsupp}) should shape the baseline to implement models and simulation methods in South-America. In this way, this chapter intended explore the potential impact that arise from a  inadequate spatial specification of \acl{top} population model. 


\subsection*{Methods}

The core task in this chapter should be the extension and improve of the current model implemented in Chile to include the fishery data from Argentine, and also to incorporate the findings from Chapter 2 and those from embedded project. Consistently, a series of scenarios will be designed to try mimic the complex dynamics of \acl{top} in South-America and assess the impact on state variables such as unexploited biomass and depletion level. 

The proponent of this PhD research plan already has access to the following sources:

{\small 
\begin{itemize}
	\item Chile:
	\begin{itemize}
		\item Comprehensive industrial logbooks from  longline fleet (1991-2013)
		\item Length and biological samples database from industrial longline fleet (1991-2013)
		\item Industrial longline landings (1989-2013)
		\item Industrial longline standardized CPUE (1991-2013) 
		\item Catch at age matrix from industrial longline (1991-1992, 1995-2013) --- include length at age keys
		\item Weight at age matrix from industrial longline (1991-2013)   
	\end{itemize}
	\item Argentine:
	\begin{itemize}
		\item Industrial trawl landings (1986-2012)
		\item Industrial longline landings (1991-2012)
		\item Industrial foreign landings (1987-2012)
		\item Industrial longline standardized CPUE (1993-2012) 
		\item Catch at age matrix from industrial trawl (1997-2012)		
		\item Catch at age matrix from industrial longline (2003-2012)
		\item Weight at age matrix from industrial fleets (1997-2012)  
	\end{itemize}
\end{itemize}
}

However, the use of Argentine data has constraints as was noted in the Chapter 1 (see pag.~\pageref{subsec:tconti}). Also are available the Chilean artisanal logbooks and biological samples for the last 10 years, some of which have been used to explore illegal, unreported and unregulated (IUU) fishing in the main exploitation areas \citep{jcq2013}.

The Chilean \acl{top} stock assessment have been carried out uniquely for the industrial fleet (south parallel 47$^o$S). An age-structured statistic model for both sexes that incorporates observation error in the catch-at-age, relative abundance index and landings was implemented in AD Model Builder  \citep{jcq2010}. This model should be the baseline for most of the scenarios and simulations in this chapter.



\subsection*{Outcomes intended}

Publishing type: \textit{Peer-reviewed article} \\
Target journal: \textit{ICES Journal of Marine Science (Oxford University Press, UK)} \\
Objectives addressed: \textit{Objectives 2 \& 3, see page~\pageref{subsec:objetives} }\\
Main milestone: \textit{Manuscript submission by July,  2016}


\subsection*{Threat \& Contingency}

Delays or difficulties in use of  Argentine data are dependent of an unsigned agreement between Chile and Argentine (see pag.~\pageref{subsec:tconti}). \\

